% Experiment Template
\textbf{Experiment No: 7 \hfill Date: 17/02/2025}
\addcontentsline{toc}{section}{7. Matrix Addition} 

\begin{center}
    \Large \subsection*{Matrix Addition}
\end{center}

\textbf{Aim}
\vspace{0.5cm}

Write a Java program to perform matrix addition.

\vspace{0.5cm}
\textbf{Algorithm}
\vspace{0.5cm}
\begin{verbatim}
1. Start
2. Take user input for the number of rows and columns of the matrices.
3. Create two matrices (matrix1 and matrix2) and take user input for their 
   elements.
4. Check if matrix addition is possible (both matrices must have the same 
   dimensions).
5. Perform element-wise addition and store the result in resultMatrix.
6. Display the resultant matrix.
7. Stop
\end{verbatim}

\vspace{0.5cm}
\textbf{Source Code}
\begin{lstlisting}[language=Java]
import java.util.Scanner;
public class MatrixAddition{
	public static void main(String args[]){
		Scanner s=new Scanner(System.in);
		System.out.println("Enter the number of rows and coloumns of first matrix: ");
		int m1=s.nextInt();
		int n1=s.nextInt();
		int mat1[][]=new int[m1][n1];
		System.out.println("Enter the first matrix: ");
		for(int i=0;i<m1;i++){
			for(int j=0;j<n1;j++) mat1[i][j]=s.nextInt();
		}
		System.out.println("Enter the number of rows and coloumns of second matrix: ");
                int m2=s.nextInt();
                int n2=s.nextInt();
		int mat2[][]=new int[m2][n2];
                System.out.println("Enter the second matrix: ");
                for(int i=0;i<m2;i++){
                        for(int j=0;j<n2;j++) mat2[i][j]=s.nextInt();
                }
		int sum[][]=new int[m1][n1];
		if(m1==m2 && n1==n2){
			for(int i=0;i<m1;i++){
                        	for(int j=0;j<n1;j++) sum[i][j]=mat1[i][j]+mat2[i][j];
			}
			System.out.println("Sum: ");
			for(int i=0;i<m1;i++){
                		for(int j=0;j<n1;j++) System.out.print(sum[i][j]+"\t");
				System.out.println();
			}
		}
		else System.out.println("Addition is not possible.");
	}
}
\end{lstlisting}

\vspace{0.5cm}
\textbf{Result}
\vspace{0.5cm}

The program was executed successfully. 

When 2 matrices were provided as input, the output was: sum of the matrices
\begin{verbatim}
Enter the number of rows and coloumns of first matrix: 
3	2
Enter the first matrix: 
1	2
2	3
3	4
Enter the number of rows and coloumns of second matrix: 
3	2 
Enter the second matrix: 
1	2
3	4
5	6
Sum: 
2	4	
5	7	
8	10
\end{verbatim}

