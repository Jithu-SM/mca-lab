% Experiment Template
\textbf{Experiment No: 17 \hfill Date: 17/03/2025}
\addcontentsline{toc}{section}{17. Multithreading} 

\begin{center}
    \Large \subsection*{Multithreading}
\end{center}

\textbf{Aim}
\vspace{0.5cm}

Write a Java program that defines two classes: one for generating and displaying the multiplication table of 5 and another for printing the first N prime numbers. Implement bothclasses using multithreading, demonstrating both approaches—by extending the Thread class and implementing the Runnable interface. Ensure proper thread management and synchronization if needed.

\vspace{0.5cm}
\textbf{Algorithm}
\vspace{0.5cm}
\begin{verbatim}
1. Start
2. Create a class MultiplicationTable that extends Thread.
    Override the run() method to print the multiplication table of 5 
    (from 1 to 10).
3. Create a class PrimeNumbers that implements Runnable.
    Define a method to generate the first N prime numbers.
    Implement the run() method to call this function.
4. In the main() method,
    Create an instance of MultiplicationTable and start the thread.
    Create an instance of PrimeNumbers, wrap it in a Thread object, and 
    start the thread.
5. Stop
\end{verbatim}

\vspace{0.5cm}
\textbf{Source Code}

\begin{lstlisting}[language=Java]
import java.util.Scanner;
class MultiplicationTable extends Thread{
	public void run(){
		System.out.println("\nMultiplication Table of 5:");
		for(int i=1;i<=10;i++){
			System.out.println("5 x "+i+" = "+(5*i));
			try{
				Thread.sleep(500);
			}catch(InterruptedException e){
				System.out.println("Multiplication thread interrupted.");
			}
		}
	}
}
class PrimeNumbers implements Runnable{
	private int N;
	public PrimeNumbers(int N){
		this.N=N;
	}
	public void run(){
		System.out.println("\nFirst "+N+" Prime Numbers:");
		int count=0,num=2;
		while(count<N){
			if(isPrime(num)){
				System.out.println(num+" ");
				count++;
				try{
					Thread.sleep(300);
				}catch(InterruptedException e){
					System.out.println("Prime number thread interrupted.");
				}
			}
			num++;
		}
		System.out.println();
	}
	private boolean isPrime(int num){
		if(num<2) return false;
		for(int i=2;i<=Math.sqrt(num);i++){
			if(num%i==0) return false;
		}
		return true;
	}
}
public class MultiThreadDemo{
	public static void main(String args[]){
		Scanner s=new Scanner(System.in);
		System.out.print("Enter the number of prime numbers to display: ");
		int N=s.nextInt();
		s.close();
		MultiplicationTable multiplicationThread=new MultiplicationTable();
		Thread primeThread=new Thread(new PrimeNumbers(N));
		multiplicationThread.start();
		primeThread.start();
		try{
			multiplicationThread.join();
			primeThread.join();
		}catch(InterruptedException e){
			System.out.println("Main thread interrupted.");
		}
		System.out.println("\nBoth threads have completed execution.");
	}
}
\end{lstlisting}

\vspace{0.5cm}
\textbf{Result}
\vspace{0.5cm}

The program was executed successfully. 

When the number of prime numbers to display: 10 were provided as input, the output was: mixed lines with multiplication table of 5 and first 10 prime numbers.
\begin{verbatim}
Enter the number of prime numbers to display: 10

Multiplication Table of 5:

First 10 Prime Numbers:
2 
5 x 1 = 5
3 
5 x 2 = 10
5 
7 
5 x 3 = 15
11 
5 x 4 = 20
13 
17 
5 x 5 = 25
19 
23 
5 x 6 = 30
29 
5 x 7 = 35

5 x 8 = 40
5 x 9 = 45
5 x 10 = 50

Both threads have completed execution.

\end{verbatim}

