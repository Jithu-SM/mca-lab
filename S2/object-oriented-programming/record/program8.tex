% Experiment Template
\textbf{Experiment No: 8 \hfill Date: 24/02/2025}
\addcontentsline{toc}{section}{8. Employee Search Using an Array of Objects} 

\begin{center}
    \Large \subsection*{Employee Search Using an Array of Objects}
\end{center}

\textbf{Aim}
\vspace{0.5cm}

Write a Java program to store employee details including employee number, name, and salary, and search for an employee by employee number.

\vspace{0.5cm}
\textbf{Algorithm}
\vspace{0.5cm}
\begin{verbatim}
1. Start
2. Create a class Employee with attributes:
    empNo (int)
    name (String)
    salary (double)
3. Define a constructor to initialize employee details.
4. Create a class EmployeeManagement that:
    Uses an ArrayList<Employee> to store employee details.
    Has a method addEmployee() to add employees.
    Has a method searchEmployeeByNumber(int empNo) to find an employee by
    their number.
5. In the main() method:
    Create an instance of EmployeeManagement.
    Add multiple employees to the list.
    Prompt the user to enter an employee number for search.
    Display employee details if found, otherwise print "Employee not 
    found".
6. Stop
\end{verbatim}

\vspace{0.5cm}
\textbf{Source Code}
\begin{lstlisting}[language=Java]
import java.util.Scanner;
import java.util.ArrayList;
class Employee{
	int empNumber;
	String name;
	double salary;
	public Employee(int empNumber, String name, double salary){
        	this.empNumber = empNumber;
        	this.name = name;
        	this.salary = salary;
	}
	public static void main(String args[]){
	       	Scanner s = new Scanner(System.in);
		ArrayList<Employee> employees = new ArrayList<>();
		System.out.println("Enter the number of Employees: ");
		int n=s.nextInt();
        	for (int i = 0; i < n; i++){
            		System.out.println("Enter details for Employee " + (i + 1) + ":");
            		System.out.print("Employee Number: ");
            		int empNumber = s.nextInt();
            		System.out.print("Employee Name: ");
            		String name = s.next();
            		System.out.print("Employee Salary: ");
            		double salary = s.nextDouble();
			employees.add(new Employee(empNumber, name, salary));
		}
        	System.out.print("\nEnter Employee Number to search: ");
        	int searchEmpNumber = s.nextInt();
        	boolean found = false;
        	for (Employee emp : employees) {
            		if (emp.empNumber == searchEmpNumber) {
                		System.out.println("\nEmployee Found:");
                		System.out.println("Employee Number: " + emp.empNumber);
                		System.out.println("Employee Name: " + emp.name);
                		System.out.println("Employee Salary: " + emp.salary);
                		found = true;
                		break;
            		}
        	}
        	if (!found) System.out.println("Employee with Employee Number " + searchEmpNumber + " not found.");
    }
}
\end{lstlisting}

\vspace{0.5cm}
\textbf{Result}
\vspace{0.5cm}

The program was executed successfully. 

When employee number 3 were provided as input, the output was: details of employee with employee number 3.
\begin{verbatim}
Enter the number of Employees: 
3
Enter details for Employee 1:
Employee Number: 1
Employee Name: Akhil
Employee Salary: 45000
Enter details for Employee 2:
Employee Number: 2
Employee Name: Anshul
Employee Salary: 40000
Enter details for Employee 3:
Employee Number: 3
Employee Name: Ajin
Employee Salary: 50000

Enter Employee Number to search: 3

Employee Found:
Employee Number: 3
Employee Name: Ajin
Employee Salary: 50000.0
\end{verbatim}

