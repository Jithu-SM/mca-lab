% Experiment Template
\textbf{Experiment No: 13 \hfill Date: 03/03/2025}
\addcontentsline{toc}{section}{13. Program to Manage Employee Collection} 

\begin{center}
    \Large \subsection*{Program to Manage Employee Collection}
\end{center}

\textbf{Aim}
\vspace{0.5cm}

Create a Java program to manage a collection of employees in a company. Implement an abstract class Employee with fields name (String) and salary (double), and an abstract method calculateSalary(). Create two subclasses: Manager (with a bonus field) and Developer (with an experience field), both overriding calculateSalary() to calculate the total salary. Implement an interface Benefits with a method calculateBenefits(), where Manager provides a fixed insurance benefit and Developer provides an allowance based on experience. Use polymorphism to store Employee objects in a list and display employee details and salary. Add method overloading in Manager for project assignment, where one method takes just a project name and the other takes both the project name and the number of team members.

\vspace{0.5cm}
\textbf{Algorithm}
\vspace{0.5cm}
\begin{verbatim}
1. Start
2. Define the Abstract Class Employee:
    Declare fields: name (String) and salary (double).
    Define an abstract method calculateSalary().
3. Create the Interface Benefits:
    Define an abstract method calculateBenefits().
4. Implement Manager Class (Extends Employee, Implements Benefits):
    Add a field bonus (double).
    Override calculateSalary() to include the bonus.
    Implement calculateBenefits() with a fixed insurance amount.
    Overload assignProject() method:
        One method takes a project name.
        Another takes both project name and team size.
5. Implement Developer Class (Extends Employee, Implements Benefits):
    Add a field experience (int).
    Override calculateSalary() to increase salary based on experience.
    Implement calculateBenefits() to return an experience-based allowance.
6. Store Employees in a List and Use Polymorphism to Display Data:
    Use an ArrayList<Employee> to store employees.
    Iterate through the list, displaying details using polymorphism.
7. Stop
\end{verbatim}

\vspace{0.5cm}
\textbf{Source Code}
\begin{lstlisting}[language=Java]
import java.util.ArrayList;
import java.util.List;
import java.util.Scanner;
interface Benefits{
	double calculateBenefits();
}
abstract class Employee{
	protected String name;
	protected double salary;
	Employee(String name, double salary){
		this.name=name;
		this.salary=salary;
	}
	abstract double calculateSalary();
	public void displayDetails(){
		System.out.println("\nEmployee Details:");
		System.out.println("Name: "+name);
		System.out.println("Total Salary: "+calculateSalary());
	}
}
class Manager extends Employee implements Benefits{
	private double bonus;
	Manager(String name,double salary,double bonus){
		super(name,salary);
		this.bonus=bonus;
	}
	double calculateSalary(){
		return salary+bonus;
	}
	public double calculateBenefits(){
		return 5000;
	}
	public void assignProject(String projectName){
		System.out.println(name+" is assigned to project: "+projectName);
	}
	public void assignProject(String projectName,int teamMembers){
		System.out.println(name+" is assigned to project: "+projectName+" with "+teamMembers+" team members.");
	}
}
class Developer extends Employee implements Benefits{
	private int experience;
	Developer(String name,double salary,int experience){
		super(name,salary);
		this.experience=experience;
	}
	double calculateSalary(){
		return salary+(experience*1000);
	}
	public double calculateBenefits(){
		return experience*500;
	}
}
public class EmployeeManagement{
	public static void main(String args[]){
		Scanner s=new Scanner(System.in);
		List<Employee> employees=new ArrayList<>();
		int choice;
		do{
			System.out.println("\nMenu:");
			System.out.println("1. Add Manager");
			System.out.println("2. Add Developer");
			System.out.println("3. Display Employees");
			System.out.println("4. Assign Project to Manager");
			System.out.println("5. Exit");
			System.out.print("Enter your choice: ");
			choice=s.nextInt();
			switch(choice){
				case 1:System.out.print("Enter Manager Name: ");
					String mName=s.next();
					System.out.print("Enter Salary: ");
					double mSalary=s.nextDouble();
					System.out.print("Enter Bonus: ");
					double mBonus=s.nextDouble();
					employees.add(new Manager(mName,mSalary,mBonus));
					System.out.println("Manager added successfully!");
					break;
				case 2:System.out.print("Enter Developer Name: ");
					String dName=s.next();
					System.out.print("Enter Salary: ");
					double dSalary=s.nextDouble();
					System.out.print("Enter Experience (years): ");
					int dExperience=s.nextInt();
					employees.add(new Developer(dName,dSalary,dExperience));
					System.out.println("Developer added successfully!");
					break;
				case 3:if(employees.isEmpty()) System.out.println("No employees to display.");
					else{
						for(Employee emp:employees){
							emp.displayDetails();
							if(emp instanceof Benefits) System.out.println("Benefits: "+((Benefits) emp).calculateBenefits());
							System.out.println();
						}
					}
					break;
				case 4:System.out.print("Enter Manager Name: ");
					String managerName=s.next();
					boolean found=false;
					for(Employee emp:employees){
						if(emp instanceof Manager && emp.name.equals(managerName)) {
							Manager manager=(Manager) emp;
							System.out.print("Enter Project Name: ");
							String projectName=s.next();
							System.out.print("Enter Number of Team Members (0 if not applicable): ");
							int teamMembers=s.nextInt();
							if(teamMembers>0) manager.assignProject(projectName,teamMembers);
							else manager.assignProject(projectName);
							found=true;
							break;
						}
					}
					if(!found) System.out.println("Manager not found!");
					break;
				case 5:System.out.println("Exiting program...");
					break;
				default:System.out.println("Invalid choice! Try again.");
			}
		}while(choice!=5);
	}
}
\end{lstlisting}

\vspace{0.5cm}
\textbf{Result}
\vspace{0.5cm}

The program was executed successfully. 

When the manager and developer details were provided as input, the output was: employee details.
\begin{verbatim}
Menu:
1. Add Manager
2. Add Developer
3. Display Employees
4. Assign Project to Manager
5. Exit
Enter your choice: 1
Enter Manager Name: Anshul
Enter Salary: 50000
Enter Bonus: 5000
Manager added successfully!

Menu:
1. Add Manager
2. Add Developer
3. Display Employees
4. Assign Project to Manager
5. Exit
Enter your choice: 2
Enter Developer Name: Akhil
Enter Salary: 30000
Enter Experience (years): 6
Developer added successfully!

Menu:
1. Add Manager
2. Add Developer
3. Display Employees
4. Assign Project to Manager
5. Exit
Enter your choice: 4
Enter Manager Name: Anshul
Enter Project Name: Java
Enter Number of Team Members (0 if not applicable): 1
Anshul is assigned to project: Java with 1 team members.

Menu:
1. Add Manager
2. Add Developer
3. Display Employees
4. Assign Project to Manager
5. Exit
Enter your choice: 3

Employee Details:
Name: Anshul
Total Salary: 55000.0
Benefits: 5000.0


Employee Details:
Name: Akhil
Total Salary: 36000.0
Benefits: 3000.0


Menu:
1. Add Manager
2. Add Developer
3. Display Employees
4. Assign Project to Manager
5. Exit
Enter your choice: 5
Exiting program...

\end{verbatim}

