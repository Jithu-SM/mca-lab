% Experiment Template
\textbf{Experiment No: 3 \hfill Date: 10/02/2025}
\addcontentsline{toc}{section}{3. Factorial of a Number} 

\begin{center}
    \Large \subsection*{Factorial of a Number}
\end{center}

\textbf{Aim}
\vspace{0.5cm}

Write a Java program to compute the factorial of a given number.

\vspace{0.5cm}
\textbf{Algorithm}
\vspace{0.5cm}
\begin{verbatim}
1. Start
2. Take an integer as input from the user.
3. Compute the factorial using either a for loop or a while loop.
4. Print the result.
5. Stop
\end{verbatim}

\vspace{0.5cm}
\textbf{Source Code}
\begin{lstlisting}[language=Java]
import java.util.Scanner;
public class Factorial{
        public static void main(String args[]){
                Scanner s=new Scanner(System.in);
                int fact=1,n;
                System.out.print("Enter a number: ");
                n=s.nextInt();
                for(int i=1;i<=n;i++) fact*=i;
                System.out.println(n+"!="+fact);
        }
}
\end{lstlisting}

\vspace{0.5cm}
\textbf{Result}
\vspace{0.5cm}

The program was executed successfully. 

When the input "5" was provided, the output was: "5!=120"
\begin{verbatim}
Enter a number: 5
5!=120
\end{verbatim}

