% Experiment Template
\textbf{Experiment No: 4 \hfill Date: 10/02/2025}
\addcontentsline{toc}{section}{4. Assigning Grades Based on Numeric Score} 

\begin{center}
    \Large \subsection*{Assigning Grades Based on Numeric Score}
\end{center}

\textbf{Aim}
\vspace{0.5cm}

Write a Java program that assigns a grade based on a numeric score.

\vspace{0.5cm}
\textbf{Algorithm}
\vspace{0.5cm}
\begin{verbatim}
1. Start
2. Take a numeric score (0–100) as input from the user.
3. Use either an if-else if-else structure or a switch-case statement to 
   assign a
    grade:
        • 90-100 → A
        • 80-89 → B
        • 70-79 → C
        • 60-69 → D
        • Below 60 → F
4. Print the assigned grade.
5. Stop
\end{verbatim}

\vspace{0.5cm}
\textbf{Source Code}
\begin{lstlisting}[language=Java]
import java.util.Scanner;
public class Grades{
        public static void main(String args[]){
                Scanner s=new Scanner(System.in);
                System.out.print("Enter score:");
                int score=s.nextInt();
                if(score>=90) System.out.println("Grade : A");
                else if(score>=80) System.out.println("Grade : B");
                else if(score>=70) System.out.println("Grade : C");
                else if(score>=60) System.out.println("Grade : D");
                else if(score<60) System.out.println("Grade : F");
                else System.out.println("Invalid Score");
        }
}
\end{lstlisting}

\vspace{0.5cm}
\textbf{Result}
\vspace{0.5cm}

The program was executed successfully. 

When the input "89" was provided, the output was: "Grade : B"
\begin{verbatim}
Enter score:89 
Grade : B
\end{verbatim}

