% Experiment Template
\textbf{Experiment No: 11 \hfill Date: 24/02/2025}
\addcontentsline{toc}{section}{11. Inheritance in Java} 

\begin{center}
    \Large \subsection*{Inheritance in Java}
\end{center}

\textbf{Aim}
\vspace{0.5cm}

Write a Java program to implement hierarchical inheritance for a book management system. Define a base class ‘Publisher‘, a derived class ‘Book‘, and two subclasses ‘Literature‘ and ‘Fiction‘. Include methods to read and display book details and demonstrate the functionality using user input.

\vspace{0.5cm}
\textbf{Algorithm}
\vspace{0.5cm}
\begin{verbatim}
1. Start
2. Define the base class Publisher
    Declare a variable publisherName.
    Create a constructor to initialize publisherName.
    Define a method displayPublisher() to print publisher details.
3. Define the derived class Book (inherits from Publisher)
    Declare variables title, author, and price.
    Create a constructor to initialize these attributes and call the 
    Publisher constructor.
    Define a method displayBook() to print book details.
4. Define the subclass Literature (inherits from Book)
    Declare a variable genre.
    Create a constructor to initialize all attributes and call the Book 
    constructor.
    Define a method displayDetails() to print literature book details.
5. Define the subclass Fiction (inherits from Book)
    Declare a variable category.
    Create a constructor to initialize all attributes and call the Book 
    constructor.
    Define a method displayDetails() to print fiction book details.
6. In the main method:
    Accept input for a Literature book (Publisher, Title, Author, Price, 
    Genre).
    Accept input for a Fiction book (Publisher, Title, Author, Price, 
    Category).
    Create objects of Literature and Fiction.
    Call their respective displayDetails() methods to print the details.
7. Stop
\end{verbatim}

\vspace{0.5cm}
\textbf{Source Code}
\begin{lstlisting}[language=Java]
import java.util.Scanner;
class Publisher{
    	String pname;
    	public Publisher(String pname){
        	this.pname = pname;
    	}
    	void display(){
        	System.out.println("Publisher Name: " + pname);
    	}
}
class Book extends Publisher{
    	String title;
    	String author;
    	double price;
    	public Book(String pname, String title, String author, double price) {
        	super(pname);
        	this.title = title;
        	this.author = author;
        	this.price = price;
    	}
    	void display(){
        	super.display();
        	System.out.println("Book Title: " + title);
        	System.out.println("Author: " + author);
        	System.out.println("Price: " + price);
    	}
}
class Literature extends Book{
    	String genre;
    	public Literature(String pname, String title, String author, double price, String genre){
        	super(pname, title, author, price);
        	this.genre = genre;
    	}
    	void display(){
        	super.display();
        	System.out.println("Genre: " + genre);
    	}
}
class Fiction extends Book{
    	String category;
    	public Fiction(String pname, String title, String author, double price, String category){
        	super(pname, title, author, price);
        	this.category = category;
    	}
    	void display(){
        	super.display();
        	System.out.println("Category: " + category);
    	}
}
public class BookManagement{
    	public static void main(String args[]){
        	Scanner s = new Scanner(System.in);
        	System.out.println("Enter details for Literature Book:");
        	System.out.print("Enter publisher name: ");
        	String pname = s.next();
        	System.out.print("Enter book title: ");
        	String title = s.next();
        	System.out.print("Enter author name: ");
        	String author = s.next();
        	System.out.print("Enter book price: ");
        	double price = s.nextDouble();
        	System.out.print("Enter genre of the book: ");
        	String genre = s.next();
        	Literature literatureBook = new Literature(pname, title, author, price, genre);
        	System.out.println("\nLiterature Book Details:");
        	literatureBook.display();
        	System.out.println("\nEnter details for Fiction Book:");
        	System.out.print("Enter publisher name: ");
        	pname = s.next();
        	System.out.print("Enter book title: ");
        	title = s.next();
        	System.out.print("Enter author name: ");
        	author = s.next();
        	System.out.print("Enter book price: ");
        	price = s.nextDouble();
        	System.out.print("Enter category of the book: ");
        	String category = s.next();
        	Fiction fictionBook = new Fiction(pname, title, author, price, category);
        	System.out.println("\nFiction Book Details:");
        	fictionBook.display();
    	}
}
\end{lstlisting}

\vspace{0.5cm}
\textbf{Result}
\vspace{0.5cm}

The program was executed successfully. 

When details of a Literature book and a Fiction book were provided as input, the output was: Details of Literature and Fiction book details.
\begin{verbatim}
Enter details for Literature Book:
Enter publisher name: ABC
Enter book title: Java
Enter author name: Akhil
Enter book price: 49.99
Enter genre of the book: Action     

Literature Book Details:
Publisher Name: ABC
Book Title: Java
Author: Akhil
Price: 49.99
Genre: Action

Enter details for Fiction Book:
Enter publisher name: OOP
Enter book title: OOPs
Enter author name: Anshul 
Enter book price: 59.99
Enter category of the book: Fantasy

Fiction Book Details:
Publisher Name: OOP
Book Title: OOPs
Author: Anshul
Price: 59.99
Category: Fantasy
\end{verbatim}

