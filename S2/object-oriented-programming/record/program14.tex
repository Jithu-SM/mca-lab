% Experiment Template
\textbf{Experiment No: 14 \hfill Date: 03/03/2025}
\addcontentsline{toc}{section}{14. Graphics Package for Geometric Figures} 

\begin{center}
    \Large \subsection*{Graphics Package for Geometric Figures}
\end{center}

\textbf{Aim}
\vspace{0.5cm}

Create a Graphics package that contains classes and interfaces for geometric figures such as ‘Rectangle‘, ‘Triangle‘, ‘Square‘, and ‘Circle‘. Test the package by finding the area of these figures.

\vspace{0.5cm}
\textbf{Algorithm}
\vspace{0.5cm}
\begin{verbatim}
1. Start
2. Create a Package graphics
    Define an interface Shape with an abstract method area().
    Implement classes: Rectangle, Triangle, Square, and Circle, each 
    implementing Shape and overriding area().
3. Test the Package in a Separate Class
    Import graphics package.
    Create objects for each shape and calculate their areas.
4. Stop
\end{verbatim}

\vspace{0.5cm}
\textbf{Source Code}

graphics/Shape.java
\begin{lstlisting}[language=Java]
package Graphics;
public interface Shape{
	double area();
}
\end{lstlisting}

graphics/Circle.java
\begin{lstlisting}[language=Java]
package Graphics;
public class Circle implements Shape{
	private double radius;
	public Circle(double radius){
		this.radius=radius;
	}
	public double area(){
		return Math.PI*radius*radius;
	}
}
\end{lstlisting}

graphics/Rectangle.java
\begin{lstlisting}[language=Java]
package Graphics;
public class Rectangle implements Shape{
	private double length,width;
	public Rectangle(double length,double width){
		this.length=length;
		this.width=width;
	}
	public double area(){
		return length*width;
	}
}
\end{lstlisting}

graphics/Square.java
\begin{lstlisting}[language=Java]
package Graphics;
public class Square implements Shape{
	private double side;
	public Square(double side){
		this.side=side;
	}
	public double area(){
		return side*side;
	}
}
\end{lstlisting}

graphics/Triangle.java
\begin{lstlisting}[language=Java]
package Graphics;
public class Triangle implements Shape{
	private double base,height;
	public Triangle(double base,double height){
		this.base=base;
		this.height=height;
	}
	public double area(){
		return 0.5*base*height;
	}
}
\end{lstlisting}

GraphicsPackage.java
\begin{lstlisting}[language=Java]
import Graphics.*;
import java.util.Scanner;
public class GraphicsPackage{
	public static void main(String args[]){
		Scanner s=new Scanner(System.in);
		int choice;
		do{
			System.out.println("\nMenu:");
			System.out.println("1. Rectangle");
			System.out.println("2. Triangle");
			System.out.println("3. Square");
			System.out.println("4. Circle");
			System.out.println("5. Exit");
			System.out.print("Enter your choice: ");
			choice=s.nextInt();
			switch(choice){
				case 1:System.out.print("Enter length: ");
					double length=s.nextDouble();
					System.out.print("Enter width: ");
					double width=s.nextDouble();
					Rectangle rectangle=new Rectangle(length, width);
					System.out.println("Area: "+rectangle.area());
					break;
				case 2:System.out.print("Enter base: ");
					double base=s.nextDouble();
					System.out.print("Enter height: ");
					double height=s.nextDouble();
					Triangle triangle=new Triangle(base, height);
					System.out.println("Area: "+triangle.area());
					break;
				case 3:System.out.print("Enter side: ");
					double side=s.nextDouble();
					Square square=new Square(side);
					System.out.println("Area: "+square.area());
					break;
				case 4:System.out.print("Enter radius: ");
					double radius=s.nextDouble();
					Circle circle=new Circle(radius);
					System.out.println("Area: "+circle.area());
					break;
				case 5:System.out.println("Exiting program...");
					break;
				default:System.out.println("Invalid choice! Try again.");
			}
		}while(choice!=5);
	}
}
\end{lstlisting}

\vspace{0.5cm}
\textbf{Result}
\vspace{0.5cm}

The program was executed successfully. 

When the shape is selected and corresponding data were provided as input, the output was: the area of the shape selected.
\begin{verbatim}
Menu:
1. Rectangle
2. Triangle
3. Square
4. Circle
5. Exit
Enter your choice: 1
Enter length: 4
Enter width: 5
Area: 20.0

Menu:
1. Rectangle
2. Triangle
3. Square
4. Circle
5. Exit
Enter your choice: 2
Enter base: 4
Enter height: 5
Area: 10.0

Menu:
1. Rectangle
2. Triangle
3. Square
4. Circle
5. Exit
Enter your choice: 3
Enter side: 4
Area: 16.0

Menu:
1. Rectangle
2. Triangle
3. Square
4. Circle
5. Exit
Enter your choice: 4
Enter radius: 5
Area: 78.53981633974483

Menu:
1. Rectangle
2. Triangle
3. Square
4. Circle
5. Exit
Enter your choice: 5
Exiting program...

\end{verbatim}

