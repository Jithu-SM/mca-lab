% Experiment Template
\textbf{Experiment No: 2 \hfill Date: 10/02/2025}
\addcontentsline{toc}{section}{2. Sum of First n Natural Numbers} 

\begin{center}
    \Large \subsection*{Sum of First n Natural Numbers}
\end{center}

\textbf{Aim}
\vspace{0.5cm}

Write a Java program to compute the sum of the first n natural numbers.

\vspace{0.5cm}
\textbf{Algorithm}
\vspace{0.5cm}
\begin{verbatim}
1. Start
2. Take an integer n as input from the user.
3. Use either a for loop or a while loop to compute the sum.
4. Print the result.
5. Stop
\end{verbatim}

\vspace{0.5cm}
\textbf{Source Code}
\begin{lstlisting}[language=Java]
import java.util.Scanner;
public class SumNatural{
        public static void main(String args[]){
                Scanner s=new Scanner(System.in);
                System.out.print("Enter a number: ");
                int sum=0,n;
                n=s.nextInt();
                for(int i=1;i<=n;i++) sum+=i;
                System.out.println("Sum of first "+n+" number is "+sum);
        }
}
\end{lstlisting}

\vspace{0.5cm}
\textbf{Result}
\vspace{0.5cm}

The program was executed successfully. 

When the input "10" was provided, the output was: "Sum of first 10 number is 55"
\begin{verbatim}
Enter a number: 10
Sum of first 10 number is 55
\end{verbatim}

