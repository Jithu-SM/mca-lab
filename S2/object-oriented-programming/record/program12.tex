% Experiment Template
\textbf{Experiment No: 12 \hfill Date: 03/03/2025}
\addcontentsline{toc}{section}{12. Calculate Area and Perimeter Using Interfaces} 

\begin{center}
    \Large \subsection*{Calculate Area and Perimeter Using Interfaces}
\end{center}

\textbf{Aim}
\vspace{0.5cm}

Write a Java Program to create an interface having prototypes of functions ‘area()‘ and ‘perimeter()‘. Create two classes ‘Circle‘ and ‘Rectangle‘ which implement the above interface. Develop a menu-driven program to find the area and perimeter of these shapes.

\vspace{0.5cm}
\textbf{Algorithm}
\vspace{0.5cm}
\begin{verbatim}
1. Start
2. Create an interface Shape with method prototypes for area() and 
   perimeter().
3. Create a Circle class that implements Shape.
    Implement the area() and perimeter() methods for a circle using the 
    formulae:
\end{verbatim}
\[
\text{Area} = \pi r^2
\]
\[
\text{Perimeter} = 2\pi r
\]
\begin{verbatim}
4. Create a Rectangle class that implements Shape.
    Implement the area() and perimeter() methods using the formulae:
\end{verbatim}
\[
\text{Area} = l \times w
\]
\[
\text{Perimeter} = 2(l + w)
\]
\begin{verbatim}
5. Develop a Menu-Driven Program:
    Use a loop to repeatedly display options to the user:
        Find area and perimeter of a Circle
        Find area and perimeter of a Rectangle
        Exit
    Based on user input, prompt for required values (radius, length, 
    breadth) 
    and display the computed area and perimeter.
    Continue until the user chooses to exit.
6. Stop
\end{verbatim}

\vspace{0.5cm}
\textbf{Source Code}
\begin{lstlisting}[language=Java]
import java.util.Scanner;
interface Shape{
	double area();
	double perimeter();
}
class Circle implements Shape{
	private double radius;
	Circle(double radius){
		this.radius=radius;
	}
	public double area(){
		return Math.PI*radius*radius;
	}
	public double perimeter(){
		return 2*Math.PI*radius;
	}
}
class Rectangle implements Shape{
	private double length,width;
	Rectangle(double length,double width){
		this.length=length;
		this.width=width;
	}
	public double area(){
		return length*width;
	}
	public double perimeter(){
		return 2*(length+width);
	}
}
public class ShapeCalculator{
	public static void main(String args[]){
		Scanner s=new Scanner(System.in);
		int choice;
		do{
			System.out.println("Menu:\n1. Circle \n2. Rectangle \n3. Exit \nEnter your choice:");
			choice=s.nextInt();
			switch(choice){
				case 1:System.out.print("Enter radius: ");
					double radius=s.nextDouble();
					Circle circle=new Circle(radius);
					System.out.println("Area: "+circle.area());
					System.out.println("Perimeter: "+circle.perimeter());
					break;
				case 2:System.out.print("Enter length: ");
					double length=s.nextDouble();
					System.out.print("Enter width: ");
					double width=s.nextDouble();
					Rectangle rectangle=new Rectangle(length,width);
					System.out.println("Area: "+rectangle.area());
					System.out.println("Perimeter: "+rectangle.perimeter());
					break;
				case 3:System.out.println("Exiting program...");
					break;
				default:System.out.println("Invalid choice!");
			}
		}while(choice!=3);
	}
}
\end{lstlisting}

\vspace{0.5cm}
\textbf{Result}
\vspace{0.5cm}

The program was executed successfully. 

When the circle radius or rectangle length and width were provided as input, the output was: corresponding area and perimeter of the shape selected.
\begin{verbatim}
Menu:
1. Circle 
2. Rectangle 
3. Exit 
Enter your choice:
1
Enter radius: 5
Area: 78.53981633974483
Perimeter: 31.41592653589793
Menu:
1. Circle 
2. Rectangle 
3. Exit 
Enter your choice:
2
Enter length: 4
Enter width: 5
Area: 20.0
Perimeter: 18.0
Menu:
1. Circle 
2. Rectangle 
3. Exit 
Enter your choice:
3
Exiting program...
\end{verbatim}

