% Experiment Template
\textbf{Experiment No: 1 \hfill Date: 10/02/2025}
\addcontentsline{toc}{section}{1. Even-Odd Classification} 

\begin{center}
    \Large \subsection*{Even-Odd Classification}
\end{center}

\textbf{Aim}
\vspace{0.5cm}

Write a Java program to check whether an input number is even or odd.

\vspace{0.5cm}
\textbf{Algorithm}
\vspace{0.5cm}
\begin{verbatim}
1. Start
2. Take an integer as input from the user.
3. Use an if-else statement to check if the number is even or odd.
4. Print the result accordingly.
5. Stop
\end{verbatim}

\vspace{0.5cm}
\textbf{Source Code}
\begin{lstlisting}[language=Java]
import java.util.Scanner;
public class EvenOdd{
        public static void main(String args[]){
                Scanner s=new Scanner(System.in);
                System.out.print("Enter a number: ");
                int n=s.nextInt();
                if(n%2==0) System.out.println(n+" is an even number");
                else System.out.println(n+" is an odd number");
        }
}
\end{lstlisting}

\vspace{0.5cm}
\textbf{Result}
\vspace{0.5cm}

The program was executed successfully. 

When the input "567" was provided, the output was: "567 is an odd number"
\begin{verbatim}
Enter a number: 567
567 is an odd number
\end{verbatim}

